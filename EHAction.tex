\documentclass[UTF8]{article}
\usepackage{ctex}
\usepackage{geometry}
\usepackage{amsmath}%align	pmatrix
\usepackage{hyperref}
%\usepackage{cancel}
\usepackage{tikz}
\title{\kaishu \\\small\today}
\date{}
\author{}
\geometry{a4paper,scale=0.8}
%\newcommand{\red}[1]{{\color{red}{#1}}}
\newcommand{\dd}{\mathrm{d}}
\newcommand{\pdo}[2]{\frac{\partial #1}{\partial #2}}%偏导,指定两个参数,定义表达式

\newcommand{\red}[1]{{\color{red}{#1}}}
\newcommand{\green}[1]{{\color{green}{#1}}}
\newcommand{\blue}[1]{{\color{blue}{#1}}}
\begin{document}
	%	\maketitle
	%		\kaishu	
\section{Einstein-Hillbert Action}
	
场和物质的作用量是分开的,先考虑没有物质的场作用量,称为爱因斯坦希尔伯特作用量。

\begin{equation}\label{action}
	S=\frac{1}{2k_D^2}\int\dd^4x\sqrt{-g}R=\frac{1}{2k_D^2}\int\dd^4x\sqrt{-g}g^{\mu\nu}R_{\mu\nu}
\end{equation}

对它变分,得到三项:
\begin{align}
\delta(\sqrt{-g}g^{\mu\nu} R_{\mu\nu})&=\delta\sqrt{-g}g^{\mu\nu} R_{\mu\nu}+\sqrt{-g}\delta g^{\mu\nu} R_{\mu\nu}+\sqrt{-g}g^{\mu\nu} \delta R_{\mu\nu}\\
&=-\frac{1}{2}g_{\mu\nu}R\sqrt{-g}\delta g^{\mu\nu}+ R_{\mu\nu}\sqrt{-g}\delta g^{\mu\nu}+\sqrt{-g}g^{\mu\nu} \delta R_{\mu\nu}\\
&=\sqrt{-g}(R_{\mu\nu}-\frac{1}{2}Rg_{\mu\nu})\delta g^{\mu\nu}-\sqrt{-g}\nabla_{\sigma}(g^{\mu\nu}\delta{\Gamma^\sigma}_{\mu\nu}-g^{\sigma\nu}\delta{\Gamma^\rho}_{\rho\nu})
\end{align}
前两项我们要把它化作$ \delta g^{\mu\nu} $的因式,最后一项化作全微分,然后用stroke定律化成边界项消去。

\textbf{引理1}$\quad  \delta g^{\mu\nu}=-g^{\mu\rho}g^{\nu\sigma}\delta g_{\rho\sigma}$,取恒等式$ g_{\rho\mu}g^{\mu\nu}={\delta^\nu}_{\rho} $变分

\textbf{引理2}$ \quad \delta\sqrt{-g}=-\frac{1}{2}\sqrt{-g}g_{\mu\nu}\delta g^{\mu\nu} $

行列式变分。对于可对角化的矩阵来说:\red{$ \log\det A=tr \log A  $}对它变分(梁灿彬pdf842,卷一81)

\begin{equation}
	\frac{1}{\det A}\delta(\det A)=tr (A^{-1}\delta A)
\end{equation}
结合这个式子和引理1即可

~\\

根据引理2,第一项

\begin{equation}
	 (\delta \sqrt{-g})g^{\mu\nu}R_{\mu\nu}=(-\frac{1}{2}\sqrt{-g}g_{\mu\nu}\delta g^{\mu\nu} )g^{\mu\nu}R_{\mu\nu}=-\frac{1}{2}\sqrt{-g}Rg_{\mu\nu}\delta g^{\mu\nu} \label{2}
\end{equation}
\textbf{第三项} 涉及对里奇张量的变分,里奇张量由曲率张量降指标而成,曲率张量又与联络有关,联络最终由度规写出。我们最终只要把它写作协变散度即可。

首先考虑曲率张量为:
\begin{equation}
	\begin{aligned}
		R_{\mu \nu \sigma}^{\rho} &=-2 \partial_{[\mu} {\Gamma^{\rho}}_{\nu]\sigma}+2 {\Gamma^{\lambda}}_{\sigma[\mu} {\Gamma^{\rho} }_{\nu] \lambda}\\
		&={\Gamma^\rho}_{\mu \sigma, v}-{\Gamma^\rho}_{v \sigma, \mu}+{\Gamma^\lambda}_{\sigma \mu}{\Gamma^{\rho}}_{\nu \lambda}-{\Gamma^{\lambda}}_{\sigma \nu} {\Gamma^\rho}_{\mu \lambda}
	\end{aligned}
\end{equation}

对直接对它变分就可以得到:

\begin{align}
\delta {R_{\mu\nu\sigma}}^\rho&=
\partial_\nu{\delta\Gamma^\rho}_{\mu \sigma}
-\partial_\mu{\delta\Gamma^\rho}_{\nu \sigma}
+{\delta\Gamma^\lambda}_{\sigma \mu}{\Gamma^{\rho}}_{\nu \lambda}
+{\Gamma^\lambda}_{\sigma \mu}{\delta\Gamma^{\rho}}_{\nu \lambda}
-{\delta\Gamma^{\lambda}}_{\sigma \nu} {\Gamma^\rho}_{\mu \lambda}
-{\Gamma^{\lambda}}_{\sigma \nu} {\delta\Gamma^\rho}_{\mu \lambda}\\
&
=\partial_\nu{\delta\Gamma^\rho}_{\mu \sigma}
+{\Gamma^{\rho}}_{\nu \lambda}{\delta\Gamma^\lambda}_{\sigma \mu}
-{\Gamma^{\lambda}}_{\sigma \nu} {\delta\Gamma^\rho}_{\mu \lambda}
-\blue{{\Gamma^\lambda}_{\mu\nu}\delta{\Gamma^\rho}_{\lambda\sigma}}
\\&
-(\partial_\mu{\delta\Gamma^\rho}_{\nu \sigma}
+{\Gamma^\rho}_{\mu \lambda}{\delta\Gamma^{\lambda}}_{\sigma \nu}
-{\Gamma^\lambda}_{\sigma \mu}{\delta\Gamma^{\rho}}_{\nu \lambda}
-\blue{{\Gamma^\lambda}_{\mu\nu}\delta{\Gamma^\rho}_{\lambda\sigma}}
)
\end{align}
(?克氏符变分表示两个协变导数之差,是张量,所以总可以用协变导数作用。)
我们额外凑了蓝色的两项,注意到一个协变倒数和普通导数之差会出现各种联络系数。
\begin{equation}
	\nabla_{\lambda}\left(\delta \Gamma_{\nu \mu}^{\rho}\right)=\partial_{\lambda}\left(\delta \Gamma_{\nu \mu}^{\rho}\right)+\Gamma_{\lambda \sigma}^{\rho} \delta \Gamma_{v \mu}^{\sigma}-\Gamma_{\lambda v}^{\sigma} \delta \Gamma_{\sigma \mu}^{\rho}-\Gamma_{\lambda \mu}^{\sigma} \delta \Gamma_{v \sigma}^{\rho}
\end{equation}

所以我们发现这两项最后恰好等于(?不能理解这是怎么想到的,但总可以验证这是对的)

\begin{equation}
\delta {R_{\mu\nu\sigma}}^\rho=\nabla_\nu\delta{\Gamma^\rho}_{\mu\sigma}-\nabla_\sigma\delta{\Gamma^\rho}_{\mu\nu}
\end{equation}
曲率张量上指标和下2指标缩并得到里奇张量,所以
\begin{equation}
 \delta{R_{\mu\nu}}=\nabla_\sigma\delta{\Gamma^\sigma}_{\mu\nu}-\nabla_\nu\delta{\Gamma^\sigma}_{\sigma\mu}
\end{equation}

算上第三项另一个因子$ g_{\mu\nu} $所有的指标都变成哑指标,这样第三项变成:

\begin{equation}
	\sqrt{-g}g^{\mu\nu}\delta R_{\mu\nu}=\sqrt{-g}\nabla_{\sigma}(g^{\mu\nu}\delta{\Gamma^\sigma}_{\mu\nu}-g^{\sigma\nu}\delta{\Gamma^\rho}_{\rho\nu})=\sqrt{-g}\nabla_{\sigma}X^\sigma\label{deltaRmunu}
\end{equation}

所以,利用stoke定理把它转化成在边界区域的积分,我们总是可以使它在无穷远处为0,这一项就消去了。


所以,剩下就把\eqref{2}代入\eqref{action}可得
\begin{align}
	\delta S=&\frac{1}{2k_D^2}\int\dd^4x(-\frac{1}{2}\sqrt{-g}Rg_{\mu\nu}\delta g^{\mu\nu}+\sqrt{-g}R_{\mu\nu}\delta g^{\mu\nu})\\
	&=\frac{1}{2k_D^2}\int\dd^4x\sqrt{-g}(R_{\mu\nu}-\frac{1}{2}Rg_{\mu\nu})\delta g^{\mu\nu}
\end{align}

所得即真空爱因斯坦方程
\section{物质项}
	
	$S_{m}=\int \mathrm{d}^{4} x \sqrt{-g} L_{m} $

$ \delta S_{m}=\int \mathrm{d}^{4} x \delta\left(\sqrt{-g} L_{m}\right)=\int \mathrm{d}^{4} x\frac{\partial\left(\sqrt{-g} L_{m}\right)}{\partial g^{\mu \nu}} \delta g^{\mu \nu}$

定义能动张量为$ T_{\mu \nu}=-\frac{2}{\sqrt{-g}} \frac{\partial\left(\sqrt{-g} L_{m}\right)}{\partial g^{\mu \nu}} $

那么这个变分就可以写作$ \delta S_{m}=-\frac{1}{2} \int \mathrm{~d}^{4} x T_{\mu \nu} \sqrt{-g} \delta g^{\mu \nu} $

	和上述场的作用量写在一起就是
	\begin{equation}\label{key}
		S=\frac{1}{2k_D^2}\int\dd ^4 xR+\int\dd^4x\mathcal{L}
	\end{equation}
	
	
	\begin{align}
			&\frac{1}{2k_D^2}\int \dd^{4} xG_{ab}\sqrt{-g}\delta g^{ab}-\frac{1}{2k_D^2}\int\dd ^4 xk_D^2
			\left(
			-\frac{2}{\sqrt{-g}} \frac{\partial\left(\sqrt{-g} L_{m}\right)}{\partial g^{ab}}
			\right) \sqrt{-g} \delta g^{ab}\\
		&=\frac{1}{2k_D^2}\int \dd^{4}\left(G_{ab}-k_D^2T_{ab}\right)\sqrt{-g} \delta g^{ab}
	\end{align}
	
	
	
	
\section{克氏符变分}
	\href{https://physics.stackexchange.com/questions/462686/whats-the-variation-of-the-christoffel-symbols-with-respect-to-the-metric}{physics stackexchange}给出了两种答案
		
	\href{https://zhuanlan.zhihu.com/p/354190513}{知乎}也有
	
	\textbf{1}
	
	The difference between two connections is a tensor, so $\delta\Gamma$ is a tensor.
	
	Evaluate your variational formula in Riemannian normal coordinates at some arbitrary point x0. Since the metric derivatives are zero at that point one gets
\begin{equation}
	\delta\Gamma^\sigma_{\mu\nu}=\frac{1}{2}\eta^{\sigma\lambda}(\partial_\nu\delta g_{\mu\lambda}+\partial_\mu\delta g_{\nu\lambda}-\partial_\lambda\delta g_{\mu\nu}),
\end{equation}
	where all functions are evaluated at $ x_0 $
	
	In Riemannian normal coordinates, $ \partial=\nabla $, so we can rewrite
\begin{equation}
	\delta\Gamma^\sigma_{\mu\nu}=\frac{1}{2}g^{\sigma\lambda}(\nabla_\nu\delta g_{\mu\lambda}+\nabla_\mu\delta g_{\nu\lambda}-\nabla_\lambda\delta g_{\mu\nu}).
\end{equation}
	
	This equation however is tensorial, so it must be valid at x0 in other coordinates too, not just Riemannian normal coordinates.
	
	Since x0 was arbitrary, this relation must then hold for any point.

\textbf{2}也可以直接强行凑出来


\begin{align}
\Gamma^{a}_{bc} &= \cfrac{1}{2}g^{ad}(\partial_{b}g_{dc} + \partial_{c}g_{bd} - \partial_{d}g_{bc})  \Rightarrow\\
\delta\Gamma^{a}_{bc} &= \cfrac{1}{2}\delta g^{ad}(\partial_{b}g_{dc} + \partial_{c}g_{bd} - \partial_{d}g_{bc})  + \cfrac{1}{2}g^{ad}(\partial_{b}\delta g_{dc} + \partial_{c}\delta g_{bd} - \partial_{d}\delta g_{bc}) \\
& = -\cfrac{1}{2}g^{ad}g^{de}(\delta g_{de})(\partial_{b}g_{dc} + \partial_{c}g_{bd} - \partial_{d}g_{bc})
+ \cfrac{1}{2}g^{ad}(\partial_{b}\delta g_{dc} + \partial_{c}\delta g_{bd} - \partial_{d}\delta g_{bc})\\
& = -g^{ad}(\delta g_{de})\Gamma^{e}_{bc}+ \cfrac{1}{2}g^{ad}(\partial_{b}\delta g_{dc} + \partial_{c}\delta g_{bd} - \partial_{d}\delta g_{bc})\\
&= \cfrac{1}{2}g^{ad}\big[\partial_{b}\delta g_{dc} + \partial_{c}\delta g_{bd} - \partial_{d}\delta g_{bc} - 2\delta g_{de}\Gamma^{e}_{bc}\big] \\
&=\cfrac{1}{2} g^{ad}\big[ \partial_{b}\delta g_{dc} -\Gamma^{e}_{bc}\delta g_{ed} -\Gamma^{e}_{bd}\delta g_{ec}\\
&+\partial_{c}\delta g_{bd} -\Gamma^{e}_{cd}\delta g_{eb} - \Gamma^{e}_{cb}\delta g_{ed} - \partial_{d}\delta g_{bc} + \Gamma^{e}_{db}\delta g_{ec} + \Gamma^{e}_{dc}\delta g_{eb} \big]\\
&= \cfrac{1}{2}g^{ad}\big[\nabla_{b}\delta g_{dc}+\nabla_{c}\delta g_{bd}- \nabla_{d}\delta g_{bc}\big]
\end{align}










\end{document}